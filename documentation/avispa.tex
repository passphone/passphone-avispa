\chapter{AVISPA}
\section{Introduction}
AVISPA stands for \textbf{A}utomated \textbf{V}alidation of \textbf{I}nternet \textbf{S}ecurity \textbf{P}rotocolls and \textbf{A}pplications and is a program to analyze cryptographic protocols.\\
Avispa translates protocols written in HLPSL (High-Level Protocol Specification Language) to the IF-language. The proofer of AVISPA are understanding this language and are interpreting it.\\
\\
In the paper \textit{\textbf{Automated Security Protocol Analysis With the AVISPA Tool}} by Luca Viganò the tool is decribed as follows:\\
\textit{The AVISPA Tool is a push-button tool for the Automated Validation of Internet Security-sensitive Protocols and Applications which rises to this challenge in a systematic way by\\
i)  providing a modular and expressive formal language for specifying security protocols nd properties, the High-Level Protocol Language HLPSL, and\\
ii) integrating different back-ends that implement variety of automatic analysis techniques ranging from protocol falsification (by finding an attack on the input protocol) to abstraction-based verification methods for infinite numbers of sessions.}\\
\\
Additionally in the paper \textit{\textbf{The AVISPA Tool for the Automated Validation of Internet Security Protocols and Applications}} by A. Armando and many others it is described as:\\
\textit{AVISPA is a push-button tool for the automated validation of Internet security-sensitive protocols and applications. It provides a modular and expressive formal language for specifying protocols and their security properties, and integrates different back-ends that implement a variety of state-of-the-art automatic analysis techniques. To the best of our knowledge, no other tool exhibits the same level of scope and robustness while enjoying the same performance and scalability.}\\
\\
The following proofer are provided:
\begin{itemize}
\item Cl-AtSe (Constraint Logic based Model Checker)
\item ofmc (On-the-fly Model-Checker)
\item satmc (SAT based Model Checker)
\item ta4sp (Tree Automata based Automatic Approximations for the Analysis of Security Protocols)
\end{itemize}

\section{Proofer}
\subsection{Cl-AtSe}
\begin{itemize}
\item bounded number of loops (bounded number of protocol steps in any trace)
\item unification of messages modulo XOR + intruder deduction rules over terms with XOR operator
\item unification of messages modulo the exponential + intruder deduction rules over terms with exponential
\item Baader Schulz unification algorithm
\end{itemize}
\subsection{ofmc}
\begin{itemize}
\item follows the dolev-yao intruder model
\item intruder can read or manipulate received messages and send them on
\item performs both protocol falsification and bounded session verification
\item guessing attacks on weak passwords
\end{itemize}
\subsection{satmc}
\begin{itemize}
\item performs both protocol falsification and bounded session verification like ofmc
\end{itemize}
\subsection{ta4sp}
\begin{itemize}
\item unbounded protocol verification by approximating the intruder knowledge by using regular tree languages and rewriting
\end{itemize}